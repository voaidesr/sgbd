\documentclass[12pt, a4paper]{report}

\usepackage[utf8]{inputenc}
\usepackage[romanian]{babel} % Sets language to Romanian
\usepackage[margin=2.5cm]{geometry} % Standard margins
\usepackage{graphicx} % For uploading diagrams/screenshots
\usepackage{float}    % To force images where you want them
\usepackage{hyperref} % For clickable Table of Contents
\usepackage{xcolor}   % For code coloring
\usepackage{listings} % For code syntax highlighting
\usepackage{titlesec} % To format chapter titles

\definecolor{codegreen}{rgb}{0,0.6,0}
\definecolor{codegray}{rgb}{0.5,0.5,0.5}
\definecolor{codepurple}{rgb}{0.58,0,0.82}
\definecolor{backcolour}{rgb}{0.95,0.95,0.92}
\definecolor{keywordcolor}{rgb}{0.0, 0.0, 1.0}

\lstdefinestyle{modernsql}{
    backgroundcolor=\color{backcolour},
    commentstyle=\color{codegreen},
    keywordstyle=\color{keywordcolor}\bfseries,
    numberstyle=\tiny\color{codegray},
    stringstyle=\color{codepurple},
    basicstyle=\ttfamily\footnotesize, % Typewriter font
    breakatwhitespace=false,
    breaklines=true,
    captionpos=b,
    keepspaces=true,
    numbers=left,
    numbersep=5pt,
    showspaces=false,
    showstringspaces=false,
    showtabs=false,
    tabsize=2,
    language=SQL,
    morekeywords={VARCHAR2, NUMBER, DATE, CONSTRAINT, PRIMARY, KEY, FOREIGN, REFERENCES, CREATE, OR, REPLACE, TRIGGER, PROCEDURE, FUNCTION, BEGIN, END, IF, THEN, ELSIF, ELSE, LOOP, EXIT, WHEN, CURSOR, OPEN, FETCH, CLOSE, TYPE, IS, RECORD, TABLE, INDEX, BY, EXCEPTION, PRAGMA, EXCEPTION_INIT}
}

\lstset{style=modernsql}

% --- Terminal Output (monospace, no syntax highlight) ---
\lstdefinestyle{terminal}{
    backgroundcolor=\color{backcolour},
    basicstyle=\ttfamily\footnotesize,
    breaklines=true,
    keepspaces=true,
    numbers=none,
    showspaces=false,
    showstringspaces=false,
    showtabs=false,
    tabsize=2,
    keywordstyle=\color{black},
    commentstyle=\color{black},
    stringstyle=\color{black}
}

\titleformat{\chapter}[block]
  {\normalfont\huge\bfseries}{\thechapter.}{1em}{\Huge}
\titlespacing*{\chapter}{0pt}{-30pt}{20pt}

\title{Proiect SGBD}
\author{Robert-Ionuț Voaideș-Negustor}
\date{2025-2026}

\begin{document}

\begin{titlepage}
    \centering
    \vspace*{1cm}
    \Huge
    \textbf{Sisteme de Gestiune a Bazelor de Date}

    \vspace{0.5cm}
    \LARGE
    Proiect

    \vspace{1.5cm}
    \textbf{Sistem de Gestiune a Restaurării Patrimoniului Cultural}

    \vfill

    \Large
    \textbf{Student:} Voaideș-Negustor Robert-Ionuț\\
    \textbf{Grupa:} 251\\
    \textbf{Seria:} 25\\
    \textbf{Anul Universitar:} 2025-2026

    \vspace{2cm}
\end{titlepage}

\tableofcontents
\newpage

\chapter*{Introducere}

\section*{Infrastructura utilizata}
\begin{itemize}
    \item \textbf{Sistem de operare:} Fedora Linux
    \item \textbf{Containerizare:} Docker
    \item \textbf{SGBD:} Oracle Database XE 21c (container \texttt{oracle-xe}, imagine \texttt{gvenzl/oracle-xe:21-slim})
    \item \textbf{Mediu de lucru:} Visual Studio Code și DataGrip
\end{itemize}

\chapter{Descrierea Bazei de Date}

\section{Scenariul Real Modelat}
Această bază de date a fost proiectată pentru a gestiona activitatea unei instituții sau companii specializate în conservarea și restaurarea patrimoniului istoric. Scenariul real modelat urmărește ciclul de viață complet al unui proiect de restaurare, de la identificarea obiectivului (monumentul) și asigurarea surselor de finanțare, până la execuția efectivă și monitorizarea calității lucrărilor.

Baza de date surprinde interacțiunile complexe dintre diversele entități implicate într-un șantier de restaurare: monumentele istorice, echipele multidisciplinare de experți, resursele materiale necesare, instituțiile statului care avizează lucrările și sursele de finanțare.

\section{Utilitatea Bazei de Date}
Implementarea acestui sistem informatic răspunde nevoii de a avea o evidență clară, transparentă și centralizată asupra modului în care este conservat patrimoniul. Principalele funcționalități și beneficii sunt:

\begin{itemize}
    \item \textbf{Trasabilitatea intervențiilor:} Sistemul permite crearea unui istoric detaliat al tuturor restaurărilor efectuate asupra unui monument. Se poate verifica oricând ce materiale s-au utilizat și ce experți au participat, informații vitale pentru viitoarele lucrări de întreținere.
    \item \textbf{Gestionarea resurselor:} Baza de date rezolvă problema alocării eficiente a resurselor. Deoarece experții și stocurile de materiale pot fi partajate între mai multe șantiere simultan, sistemul evidențiază clar distribuția acestora.
    \item \textbf{Monitorizarea financiară și legală:} Permite urmărirea bugetelor alocate din diverse surse de finanțare, precum și starea avizelor necesare de la autoritățile competente (ex: Ministerul Culturii, Primărie) și rezultatele inspecțiilor de șantier.
\end{itemize}

\newpage

\chapter{Diagrama Entitate-Relație (ERD)}

\begin{figure}[H]
    \centering
    \includegraphics[width=0.95\textwidth]{imgs/sgbd-er.jpg}
    \caption{Diagrama ERD a bazei de date}
\end{figure}

\newpage

\chapter{Diagrama Conceptuală}

\begin{figure}[H]
    \centering
    \includegraphics[width=0.95\textwidth]{imgs/sgbd-concept.drawio.png}
    \caption{Diagrama Conceptuală}
\end{figure}

\newpage

\chapter{Implementarea Structurii}
Mai jos sunt prezentate instrucțiunile de creare a tabelelor și constrângerile de integritate.

\begin{lstlisting}[caption={Crearea tabelelor}]
CREATE TABLE monument (
    id_monument NUMBER(5) CONSTRAINT pk_monument PRIMARY KEY,
    denumire    VARCHAR2(100) NOT NULL,
    locatie     VARCHAR2(100) NOT NULL
);


CREATE TABLE expert (
    id_expert    NUMBER(5) CONSTRAINT pk_expert PRIMARY KEY,
    nume         VARCHAR2(50) NOT NULL,
    specializare VARCHAR2(50) NOT NULL,
    email        VARCHAR2(100) CONSTRAINT uq_expert_email UNIQUE
);


CREATE TABLE material (
    id_material NUMBER(5) CONSTRAINT pk_material PRIMARY KEY,
    denumire    VARCHAR2(50) NOT NULL,
    clasa       VARCHAR2(20),
    risc        VARCHAR2(20)
);


CREATE TABLE finantare (
    id_finantare NUMBER(5) CONSTRAINT pk_finantare PRIMARY KEY,
    sursa        VARCHAR2(50) NOT NULL,
    buget        NUMBER(12, 2) CONSTRAINT ck_buget_pozitiv CHECK (buget > 0)
);


CREATE TABLE autoritate (
    id_autoritate NUMBER(5) CONSTRAINT pk_autoritate PRIMARY KEY,
    nume          VARCHAR2(100) NOT NULL,
    tip           VARCHAR2(50),
    atributie     VARCHAR2(100)
);

CREATE TABLE restaurare (
    id_restaurare NUMBER(5) CONSTRAINT pk_restaurare PRIMARY KEY,
    id_monument   NUMBER(5) NOT NULL,
    data_start    DATE DEFAULT SYSDATE,
    data_final    DATE,
    stadiu        VARCHAR2(20) CONSTRAINT ck_stadiu CHECK (stadiu IN ('Planificat', 'In executie', 'Finalizat', 'Suspendat')),
    CONSTRAINT fk_restaurare_monument FOREIGN KEY (id_monument) REFERENCES monument(id_monument),
    CONSTRAINT ck_date_valide CHECK (data_final >= data_start)
);


CREATE TABLE aviz (
    id_aviz       NUMBER(5) CONSTRAINT pk_aviz PRIMARY KEY,
    id_restaurare NUMBER(5) NOT NULL,
    id_autoritate NUMBER(5) NOT NULL,
    tip_aviz      VARCHAR2(50) NOT NULL,
    data_aviz     DATE DEFAULT SYSDATE,
    CONSTRAINT fk_aviz_restaurare FOREIGN KEY (id_restaurare) REFERENCES restaurare(id_restaurare),
    CONSTRAINT fk_aviz_autoritate FOREIGN KEY (id_autoritate) REFERENCES autoritate(id_autoritate)
);


CREATE TABLE inspectie (
    id_inspectie   NUMBER(5) CONSTRAINT pk_inspectie PRIMARY KEY,
    id_restaurare  NUMBER(5) NOT NULL,
    data_inspectie DATE DEFAULT SYSDATE,
    rezultat       VARCHAR2(200),
    CONSTRAINT fk_inspectie_restaurare FOREIGN KEY (id_restaurare) REFERENCES restaurare(id_restaurare)
);


CREATE TABLE expert_restaurare (
    id_expert     NUMBER(5),
    id_restaurare NUMBER(5),
    CONSTRAINT pk_expert_restaurare PRIMARY KEY (id_expert, id_restaurare),
    CONSTRAINT fk_er_expert FOREIGN KEY (id_expert) REFERENCES expert(id_expert),
    CONSTRAINT fk_er_restaurare FOREIGN KEY (id_restaurare) REFERENCES restaurare(id_restaurare)
);


CREATE TABLE material_restaurare (
    id_material   NUMBER(5),
    id_restaurare NUMBER(5),
    CONSTRAINT pk_material_restaurare PRIMARY KEY (id_material, id_restaurare),
    CONSTRAINT fk_mr_material FOREIGN KEY (id_material) REFERENCES material(id_material),
    CONSTRAINT fk_mr_restaurare FOREIGN KEY (id_restaurare) REFERENCES restaurare(id_restaurare)
);


CREATE TABLE finantare_restaurare (
    id_finantare  NUMBER(5),
    id_restaurare NUMBER(5),
    CONSTRAINT pk_finantare_restaurare PRIMARY KEY (id_finantare, id_restaurare),
    CONSTRAINT fk_fr_finantare FOREIGN KEY (id_finantare) REFERENCES finantare(id_finantare),
    CONSTRAINT fk_fr_restaurare FOREIGN KEY (id_restaurare) REFERENCES restaurare(id_restaurare)
);

\end{lstlisting}

\newpage

\textbf{Dovada rulării în Oracle:}
\begin{figure}[H]
    \centering
    \includegraphics[width=0.3\textwidth]{imgs/sgbd001.png}
    \caption{Rularea comenzilor CREATE TABLE}
\end{figure}

\newpage

\chapter{Popularea Bazei de Date}
Au fost inserate minim 5 înregistrări pentru entitățile independente și 10 pentru cele asociative.

\begin{lstlisting}[caption={Inserarea datelor}]
INSERT INTO monument VALUES (1, 'Castelul Peles', 'Sinaia');
INSERT INTO monument VALUES (2, 'Biserica Neagra', 'Brasov');
INSERT INTO monument VALUES (3, 'Cetatea de Scaun', 'Suceava');
INSERT INTO monument VALUES (4, 'Manastirea Voronet', 'Gura Humorului');
INSERT INTO monument VALUES (5, 'Cazinoul', 'Constanta');


INSERT INTO expert VALUES (101, 'Popescu Ion', 'Arhitect', 'popescu.i@expert.ro');
INSERT INTO expert VALUES (102, 'Ionescu Maria', 'Inginer Structurist', 'maria.i@expert.ro');
INSERT INTO expert VALUES (103, 'Georgescu Vlad', 'Restaurator Pictura', 'vlad.g@expert.ro');
INSERT INTO expert VALUES (104, 'Dumitru Ana', 'Arheolog', 'ana.d@expert.ro');
INSERT INTO expert VALUES (105, 'Stanescu Dan', 'Manager Proiect', 'dan.s@expert.ro');


INSERT INTO material VALUES (201, 'Piatra de rau', 'Natural', 'Scazut');
INSERT INTO material VALUES (202, 'Mortar hidraulic', 'Sintetic', 'Mediu');
INSERT INTO material VALUES (203, 'Lemn de stejar tratat', 'Natural', 'Ridicat');
INSERT INTO material VALUES (204, 'Pigment mineral', 'Chimic', 'Mediu');
INSERT INTO material VALUES (205, 'Caramida arsa', 'Compozit', 'Scazut');


INSERT INTO finantare VALUES (301, 'Ministerul Culturii', 5000000);
INSERT INTO finantare VALUES (302, 'Fonduri Europene REGIO', 12000000);
INSERT INTO finantare VALUES (303, 'Buget Local Sinaia', 200000);
INSERT INTO finantare VALUES (304, 'Donatii Private ONG', 50000);
INSERT INTO finantare VALUES (305, 'Granturi Norvegiene', 3500000);


INSERT INTO autoritate VALUES (401, 'Ministerul Culturii', 'Guvernamental', 'Avizare patrimoniu');
INSERT INTO autoritate VALUES (402, 'Primaria Sinaia', 'Local', 'Certificat Urbanism');
INSERT INTO autoritate VALUES (403, 'ISC Brasov', 'Inspectie', 'Control Calitate');
INSERT INTO autoritate VALUES (404, 'Directia Judeteana Cultura', 'Judetean', 'Monitorizare');
INSERT INTO autoritate VALUES (405, 'Primaria Constanta', 'Local', 'Autorizatie Constructie');


INSERT INTO restaurare VALUES (1001, 1, TO_DATE('01-03-2024','DD-MM-YYYY'), TO_DATE('01-12-2025','DD-MM-YYYY'), 'In executie');
INSERT INTO restaurare VALUES (1002, 2, TO_DATE('15-05-2023','DD-MM-YYYY'), TO_DATE('15-05-2024','DD-MM-YYYY'), 'Finalizat');
INSERT INTO restaurare VALUES (1003, 5, TO_DATE('01-01-2020','DD-MM-YYYY'), TO_DATE('01-01-2026','DD-MM-YYYY'), 'In executie');
INSERT INTO restaurare VALUES (1004, 3, TO_DATE('10-10-2025','DD-MM-YYYY'), NULL, 'Planificat');
INSERT INTO restaurare VALUES (1005, 4, TO_DATE('01-06-2024','DD-MM-YYYY'), TO_DATE('01-09-2024','DD-MM-YYYY'), 'Suspendat');


INSERT INTO aviz VALUES (1, 1001, 401, 'Aviz Favorabil', TO_DATE('20-02-2024','DD-MM-YYYY'));
INSERT INTO aviz VALUES (2, 1001, 402, 'Autorizatie Constructie', TO_DATE('25-02-2024','DD-MM-YYYY'));
INSERT INTO aviz VALUES (3, 1003, 405, 'Prelungire Autorizatie', TO_DATE('10-01-2024','DD-MM-YYYY'));


INSERT INTO inspectie VALUES (1, 1001, TO_DATE('01-04-2024','DD-MM-YYYY'), 'Conform cu proiectul');
INSERT INTO inspectie VALUES (2, 1003, TO_DATE('15-04-2024','DD-MM-YYYY'), 'Degradari neprevazute la fatada');


INSERT INTO expert_restaurare VALUES (101, 1001);
INSERT INTO expert_restaurare VALUES (102, 1001);
INSERT INTO expert_restaurare VALUES (105, 1001);
INSERT INTO expert_restaurare VALUES (101, 1003);
INSERT INTO expert_restaurare VALUES (102, 1003);
INSERT INTO expert_restaurare VALUES (103, 1005);
INSERT INTO expert_restaurare VALUES (104, 1004);
INSERT INTO expert_restaurare VALUES (105, 1002);
INSERT INTO expert_restaurare VALUES (102, 1002);
INSERT INTO expert_restaurare VALUES (101, 1002);


INSERT INTO material_restaurare VALUES (203, 1001);
INSERT INTO material_restaurare VALUES (205, 1001);
INSERT INTO material_restaurare VALUES (201, 1002);
INSERT INTO material_restaurare VALUES (202, 1002);
INSERT INTO material_restaurare VALUES (201, 1003);
INSERT INTO material_restaurare VALUES (202, 1003);
INSERT INTO material_restaurare VALUES (204, 1003);
INSERT INTO material_restaurare VALUES (201, 1004);
INSERT INTO material_restaurare VALUES (204, 1005);
INSERT INTO material_restaurare VALUES (202, 1005);


INSERT INTO finantare_restaurare VALUES (301, 1001);
INSERT INTO finantare_restaurare VALUES (303, 1001);
INSERT INTO finantare_restaurare VALUES (304, 1002);
INSERT INTO finantare_restaurare VALUES (305, 1002);
INSERT INTO finantare_restaurare VALUES (302, 1003);
INSERT INTO finantare_restaurare VALUES (301, 1003);
INSERT INTO finantare_restaurare VALUES (305, 1003);
INSERT INTO finantare_restaurare VALUES (302, 1004);
INSERT INTO finantare_restaurare VALUES (301, 1005);
INSERT INTO finantare_restaurare VALUES (304, 1005);


COMMIT;

\end{lstlisting}

\textbf{Dovada rulării în Oracle:}
\begin{figure}[H]
    \centering
    % \includegraphics[width=0.8\textwidth]{screenshot_dml.png}
    \caption{Rezultatul inserării datelor}
\end{figure}

\newpage

\chapter{Utilizarea Colecțiilor (Tablouri/Vectori)}
\section*{Enunțul Problemei}
[Formulați problema în limbaj natural aici...]

\section*{Rezolvare PL/SQL}
\begin{lstlisting}[caption={Subprogram cu colecții}]
CREATE OR REPLACE PROCEDURE demo_colectii IS
    TYPE t_tablou IS TABLE OF VARCHAR2(100);
    v_lista t_tablou := t_tablou();
BEGIN
    -- Codul tau aici
    NULL;
END;
/
\end{lstlisting}

\section*{Apel și Rezultat}
\begin{lstlisting}
BEGIN
    demo_colectii;
END;
/
\end{lstlisting}

\begin{figure}[H]
    \centering
    % \includegraphics[width=0.8\textwidth]{screenshot_cerinta6.png}
    \caption{Execuția cerinței 6}
\end{figure}

\newpage

\chapter{Utilizarea Cursoarelor}
\section*{Enunțul Problemei}
[Problema care necesită 2 tipuri de cursoare, unul parametrizat...]

\section*{Rezolvare PL/SQL}
\begin{lstlisting}[caption={Subprogram cu cursoare}]
-- Codul tau aici
\end{lstlisting}

\section*{Apel și Rezultat}
\begin{figure}[H]
    \centering
    % \includegraphics[width=0.8\textwidth]{screenshot_cerinta7.png}
    \caption{Execuția cerinței 7}
\end{figure}

\newpage

\chapter{Funcție Stocată}
\section*{Enunțul Problemei}
[Funcție care utilizează 3 tabele și tratează excepțiile...]

\section*{Rezolvare PL/SQL}
\begin{lstlisting}[caption={Funcția PL/SQL}]
-- Codul tau aici
\end{lstlisting}

\section*{Apel și Rezultat (Tratare Excepții)}
\begin{figure}[H]
    \centering
    % \includegraphics[width=0.8\textwidth]{screenshot_cerinta8.png}
    \caption{Apelul funcției și tratarea excepțiilor}
\end{figure}

\newpage

\chapter{Procedură Stocată}
\section*{Enunțul Problemei}
[Procedură cu 5 tabele și excepții proprii...]

\section*{Rezolvare PL/SQL}
\begin{lstlisting}[caption={Procedura PL/SQL}]
-- Codul tau aici
\end{lstlisting}

\section*{Apel și Rezultat}
\begin{figure}[H]
    \centering
    % \includegraphics[width=0.8\textwidth]{screenshot_cerinta9.png}
    \caption{Execuția procedurii}
\end{figure}

\newpage

\chapter{Trigger LMD la Nivel de Comandă}
\section*{Definire Trigger}
\begin{lstlisting}
-- Codul trigger-ului
\end{lstlisting}

\section*{Declanșare}
\begin{figure}[H]
    \centering
    % \includegraphics[width=0.8\textwidth]{screenshot_cerinta10.png}
    \caption{Demonstrație declanșare trigger}
\end{figure}

\newpage

\chapter{Trigger LMD la Nivel de Linie}
\section*{Definire Trigger}
\begin{lstlisting}
-- Codul trigger-ului
\end{lstlisting}

\section*{Declanșare}
\begin{figure}[H]
    \centering
    % \includegraphics[width=0.8\textwidth]{screenshot_cerinta11.png}
    \caption{Demonstrație declanșare trigger}
\end{figure}

\newpage

\chapter{Trigger LDD}
\section*{Definire Trigger}
\begin{lstlisting}
-- Codul trigger-ului pentru comenzi de tip CREATE/DROP/ALTER
\end{lstlisting}

\section*{Declanșare}
\begin{figure}[H]
    \centering
    % \in% =========================================================
% 3. INTRODUCERE [cite: 31, 32]
% =========================================================cludegraphics[width=0.8\textwidth]{screenshot_cerinta12.png}
    \caption{Demonstrație declanșare trigger LDD}
\end{figure}

\newpage

\chapter{Pachet PL/SQL (Opțional)}
\section*{Specificația Pachetului}
\begin{lstlisting}
-- Codul specificatiei
\end{lstlisting}

\section*{Corpul Pachetului}
\begin{lstlisting}
-- Codul body-ului
\end{lstlisting}

\section*{Apel Funcționalități}
\begin{figure}[H]
    \centering
    % \includegraphics[width=0.8\textwidth]{screenshot_cerinta13.png}
    \caption{Execuția elementelor din pachet}
\end{figure}

\end{document}
