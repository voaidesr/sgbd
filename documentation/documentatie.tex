\documentclass[12pt, a4paper]{report}

% --- Packages ---
\usepackage[utf8]{inputenc}
\usepackage[romanian]{babel} % Sets language to Romanian
\usepackage[margin=2.5cm]{geometry} % Standard margins
\usepackage{graphicx} % For uploading diagrams/screenshots
\usepackage{float}    % To force images where you want them
\usepackage{hyperref} % For clickable Table of Contents
\usepackage{xcolor}   % For code coloring
\usepackage{listings} % For code syntax highlighting
\usepackage{titlesec} % To format chapter titles

% --- Modern Code Highlighting Setup (SQL/PLSQL) ---
% Defining colors similar to VS Code / Modern Editors
\definecolor{codegreen}{rgb}{0,0.6,0}
\definecolor{codegray}{rgb}{0.5,0.5,0.5}
\definecolor{codepurple}{rgb}{0.58,0,0.82}
\definecolor{backcolour}{rgb}{0.95,0.95,0.92}
\definecolor{keywordcolor}{rgb}{0.0, 0.0, 1.0}

\lstdefinestyle{modernsql}{
    backgroundcolor=\color{backcolour},
    commentstyle=\color{codegreen},
    keywordstyle=\color{keywordcolor}\bfseries,
    numberstyle=\tiny\color{codegray},
    stringstyle=\color{codepurple},
    basicstyle=\ttfamily\footnotesize, % Typewriter font
    breakatwhitespace=false,
    breaklines=true,
    captionpos=b,
    keepspaces=true,
    numbers=left,
    numbersep=5pt,
    showspaces=false,
    showstringspaces=false,
    showtabs=false,
    tabsize=2,
    language=SQL,
    morekeywords={VARCHAR2, NUMBER, DATE, CONSTRAINT, PRIMARY, KEY, FOREIGN, REFERENCES, CREATE, OR, REPLACE, TRIGGER, PROCEDURE, FUNCTION, BEGIN, END, IF, THEN, ELSIF, ELSE, LOOP, EXIT, WHEN, CURSOR, OPEN, FETCH, CLOSE, TYPE, IS, RECORD, TABLE, INDEX, BY, EXCEPTION, PRAGMA, EXCEPTION_INIT}
}

\lstset{style=modernsql}

% --- Chapter Formatting ---
% Starts new chapters on a new page automatically
\titleformat{\chapter}[block]
  {\normalfont\huge\bfseries}{\thechapter.}{1em}{\Huge}
\titlespacing*{\chapter}{0pt}{-30pt}{20pt}

% --- Document Info ---
\title{Proiect SGBD}
\author{Numele și Prenumele Tău}
\date{2025-2026}

\begin{document}

% =========================================================
% 1. PAGINA DE TITLU [cite: 29]
% =========================================================
\begin{titlepage}
    \centering
    \vspace*{1cm}
    \Huge
    \textbf{Sisteme de Gestiune a Bazelor de Date}

    \vspace{0.5cm}
    \LARGE
    Proiect Semestrial

    \vspace{1.5cm}
    \textbf{Titlul Proiectului: [Numele Bazei Tale de Date]}

    \vfill

    \Large
    \textbf{Student:} Nume Prenume\\
    \textbf{Grupa:} [Grupa Ta]\\
    \textbf{Seria:} [Seria Ta]\\
    \textbf{Anul Universitar:} 2025-2026

    \vspace{2cm}
\end{titlepage}

% =========================================================
% 2. CUPRINS [cite: 30]
% =========================================================
\tableofcontents
\newpage

% =========================================================
% 3. INTRODUCERE [cite: 31, 32]
% =========================================================
\chapter*{Introducere}
\addcontentsline{toc}{chapter}{Introducere}

\section*{Tema Proiectului}
[Descrie aici pe scurt ce face aplicația/baza de date.]

\section*{Infrastructura Utilizată}
\begin{itemize}
    \item \textbf{SGBD:} Oracle Database [Ex: 19c / 21c Express Edition]
    \item \textbf{Sistem de Operare:} [Ex: Windows 11 / Linux Ubuntu]
    \item \textbf{Mediul de lucru:} [Ex: SQL Developer / DataGrip / Mașină Virtuală]
\end{itemize}

\newpage

% =========================================================
% CERINȚA 1: Descrierea Bazei de Date [cite: 6]
% =========================================================
\chapter{Descrierea Bazei de Date}
[Prezentați pe scurt baza de date și utilitatea ei. Descrieți scenariul real pe care îl modelează.]

\newpage

% =========================================================
% CERINȚA 2: Diagrama ERD [cite: 7]
% =========================================================
\chapter{Diagrama Entitate-Relație (ERD)}
Diagrama conține entitățile, relațiile și atributele definite în limba română.

\begin{figure}[H]
    \centering
    % \includegraphics[width=1\textwidth]{nume_imagine_erd.png}
    \caption{Diagrama ERD a bazei de date}
\end{figure}

\newpage

% =========================================================
% CERINȚA 3: Diagrama Conceptuală [cite: 8]
% =========================================================
\chapter{Diagrama Conceptuală}
Modelul propus integrând toate atributele necesare.

\begin{figure}[H]
    \centering
    % \includegraphics[width=1\textwidth]{nume_imagine_conceptuala.png}
    \caption{Diagrama Conceptuală}
\end{figure}

\newpage

% =========================================================
% CERINȚA 4: Implementarea (DDL) [cite: 9]
% =========================================================
\chapter{Implementarea Structurii (DDL)}
Mai jos sunt prezentate instrucțiunile de creare a tabelelor și constrângerile de integritate.

\begin{lstlisting}[caption={Crearea tabelelor}]
-- Exemplu de cod SQL
CREATE TABLE clienti (
    id_client NUMBER(5) PRIMARY KEY,
    nume VARCHAR2(50) NOT NULL,
    data_nasterii DATE
);

-- Adauga restul codului DDL aici
\end{lstlisting}

\textbf{Dovada rulării în Oracle:}
\begin{figure}[H]
    \centering
    % \includegraphics[width=0.8\textwidth]{screenshot_ddl.png}
    \caption{Rularea comenzilor CREATE TABLE}
\end{figure}

\newpage

% =========================================================
% CERINȚA 5: Inserarea Datelor (DML) [cite: 10]
% =========================================================
\chapter{Popularea Bazei de Date (DML)}
Au fost inserate minim 5 înregistrări pentru entitățile independente și 10 pentru cele asociative.

\begin{lstlisting}[caption={Inserarea datelor}]
INSERT INTO clienti (id_client, nume, data_nasterii)
VALUES (1, 'Popescu Ion', TO_DATE('1990-05-12', 'YYYY-MM-DD'));

-- Adauga restul insert-urilor aici
\end{lstlisting}

\textbf{Dovada rulării în Oracle:}
\begin{figure}[H]
    \centering
    % \includegraphics[width=0.8\textwidth]{screenshot_dml.png}
    \caption{Rezultatul inserării datelor}
\end{figure}

\newpage

% =========================================================
% CERINȚA 6: Colecții [cite: 11]
% =========================================================
\chapter{Utilizarea Colecțiilor (Tablouri/Vectori)}
\section*{Enunțul Problemei}
[Formulați problema în limbaj natural aici...]

\section*{Rezolvare PL/SQL}
\begin{lstlisting}[caption={Subprogram cu colecții}]
CREATE OR REPLACE PROCEDURE demo_colectii IS
    TYPE t_tablou IS TABLE OF VARCHAR2(100);
    v_lista t_tablou := t_tablou();
BEGIN
    -- Codul tau aici
    NULL;
END;
/
\end{lstlisting}

\section*{Apel și Rezultat}
\begin{lstlisting}
BEGIN
    demo_colectii;
END;
/
\end{lstlisting}

\begin{figure}[H]
    \centering
    % \includegraphics[width=0.8\textwidth]{screenshot_cerinta6.png}
    \caption{Execuția cerinței 6}
\end{figure}

\newpage

% =========================================================
% CERINȚA 7: Cursoare [cite: 13]
% =========================================================
\chapter{Utilizarea Cursoarelor}
\section*{Enunțul Problemei}
[Problema care necesită 2 tipuri de cursoare, unul parametrizat...]

\section*{Rezolvare PL/SQL}
\begin{lstlisting}[caption={Subprogram cu cursoare}]
-- Codul tau aici
\end{lstlisting}

\section*{Apel și Rezultat}
\begin{figure}[H]
    \centering
    % \includegraphics[width=0.8\textwidth]{screenshot_cerinta7.png}
    \caption{Execuția cerinței 7}
\end{figure}

\newpage

% =========================================================
% CERINȚA 8: Funcție [cite: 15, 16]
% =========================================================
\chapter{Funcție Stocată}
\section*{Enunțul Problemei}
[Funcție care utilizează 3 tabele și tratează excepțiile...]

\section*{Rezolvare PL/SQL}
\begin{lstlisting}[caption={Funcția PL/SQL}]
-- Codul tau aici
\end{lstlisting}

\section*{Apel și Rezultat (Tratare Excepții)}
\begin{figure}[H]
    \centering
    % \includegraphics[width=0.8\textwidth]{screenshot_cerinta8.png}
    \caption{Apelul funcției și tratarea excepțiilor}
\end{figure}

\newpage

% =========================================================
% CERINȚA 9: Procedură [cite: 18, 19]
% =========================================================
\chapter{Procedură Stocată}
\section*{Enunțul Problemei}
[Procedură cu 5 tabele și excepții proprii...]

\section*{Rezolvare PL/SQL}
\begin{lstlisting}[caption={Procedura PL/SQL}]
-- Codul tau aici
\end{lstlisting}

\section*{Apel și Rezultat}
\begin{figure}[H]
    \centering
    % \includegraphics[width=0.8\textwidth]{screenshot_cerinta9.png}
    \caption{Execuția procedurii}
\end{figure}

\newpage

% =========================================================
% CERINȚA 10: Trigger LMD (Comandă) [cite: 22]
% =========================================================
\chapter{Trigger LMD la Nivel de Comandă}
\section*{Definire Trigger}
\begin{lstlisting}
-- Codul trigger-ului
\end{lstlisting}

\section*{Declanșare}
\begin{figure}[H]
    \centering
    % \includegraphics[width=0.8\textwidth]{screenshot_cerinta10.png}
    \caption{Demonstrație declanșare trigger}
\end{figure}

\newpage

% =========================================================
% CERINȚA 11: Trigger LMD (Linie) [cite: 23]
% =========================================================
\chapter{Trigger LMD la Nivel de Linie}
\section*{Definire Trigger}
\begin{lstlisting}
-- Codul trigger-ului
\end{lstlisting}

\section*{Declanșare}
\begin{figure}[H]
    \centering
    % \includegraphics[width=0.8\textwidth]{screenshot_cerinta11.png}
    \caption{Demonstrație declanșare trigger}
\end{figure}

\newpage

% =========================================================
% CERINȚA 12: Trigger LDD [cite: 24]
% =========================================================
\chapter{Trigger LDD}
\section*{Definire Trigger}
\begin{lstlisting}
-- Codul trigger-ului pentru comenzi de tip CREATE/DROP/ALTER
\end{lstlisting}

\section*{Declanșare}
\begin{figure}[H]
    \centering
    % \includegraphics[width=0.8\textwidth]{screenshot_cerinta12.png}
    \caption{Demonstrație declanșare trigger LDD}
\end{figure}

\newpage

% =========================================================
% CERINȚA 13: Pachet (Opțional) [cite: 25]
% =========================================================
\chapter{Pachet PL/SQL (Opțional)}
\section*{Specificația Pachetului}
\begin{lstlisting}
-- Codul specificatiei
\end{lstlisting}

\section*{Corpul Pachetului}
\begin{lstlisting}
-- Codul body-ului
\end{lstlisting}

\section*{Apel Funcționalități}
\begin{figure}[H]
    \centering
    % \includegraphics[width=0.8\textwidth]{screenshot_cerinta13.png}
    \caption{Execuția elementelor din pachet}
\end{figure}

\end{document}
